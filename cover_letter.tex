\documentclass{article}
\usepackage{graphicx} % Required for inserting images
\usepackage{hyperref}

\begin{document}

\noindent Dear SW23 Special Issue Editors,
\\

We submit our article \textbf{Towards Sharing Tools and Artefacts for Reusable Simulations (STARS) in Healthcare}.  This is a continuation of the open science work that we published at the OR Society Simulation Workshop 2023.  In summary, the work presents a novel pilot framework for publishing and sharing Discrete-event Simulation (DES) computer models in healthcare. The STARS framework supports model reuse, and increased accessibility for DES models coded in Free and Open Source Software.

\textbf{We would like to use the optional single blind peer review for our manuscript.} The work is open science focused and has an number of external data and code archives that peer reviewers may wish to access. 

Since the March 2023 workshop we have greatly expanded the work:

\begin{enumerate}
    \item We have generalised the framework.
    \item We have modularised the framework into essential and optional components - this specifically tackles the time needed for open science activities in DES.
    \item We have added two additional, and substantial, applied examples of the framework in python that vary the technology used 
    \item We have greatly expanded the literature review and in particular describe the challenges of open science for the M\&S community.
    \item We include a detailed discussion of the strengths, currently limitations and ways forward for the framework.
\end{enumerate}

The work has the following contributions to JoS readers and the M\&S community:

\begin{itemize}
    \item The framework builds on and addresses the results of our recent open science review published in JoS that identified the current weaknesses of DES literature. \url{https://doi.org/10.1080/17477778.2023.2260772}
    \item Supports new ways to use web technology and FOSS to collaborate with busy healthcare clinicians on DES projects.
    \item Tackles the current barriers to publishing resuable models in healthcare DES.
    \item Support researchers manage their time and workflow for open science in DES projects.
    \item Transfers open science knowledge to early career researchers who stand to benefit the most from adopting open science practices.
\end{itemize}

The following papers were influential in our work and you may wish to consider one or more of the authors as potential reviewers of our submission.

\begin{itemize}
    \item G Dagkakis \& C Heavey (2016) A review of open source discrete event simulation software for operations research, Journal of Simulation, 10:3, 193-206, DOI: 10.1057/jos.2015.9 
    \item Janssen MA, Pritchard C, Lee A. On code sharing and model documentation of published individual and agent-based models. Environ Model Softw. 2020 Dec;134:104873. doi: 10.1016/j.envsoft.2020.104873. Epub 2020 Sep 16. PMID: 32958993; PMCID: PMC7493807.
\end{itemize}
 
\textbf{We note that the word count is 10,200 words}.  Given the lack of open science knowledge in modelling and simulation we included substantial background section describing literature and state-of-the-art. Readers will best placed to understand and adopt our methods if they are more familiar with this area.

We hope you agree that the work is of interest to the JoS readership.  For reference the original SW23 paper is published online \url{https://www.theorsociety.com/media/7313/doiorg1036819sw23030.pdf}.

\vspace{0.5cm}


\noindent Yours faithfully,

\vspace{0.5cm}
\noindent Dr Thomas Monks (on behalf of the study team)

\end{document}
